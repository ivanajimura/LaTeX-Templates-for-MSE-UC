\documentclass{linghandout}
\usepackage[utf8]{inputenc}

\newcommand{\sfut}{\textsc{SFut}\xspace}

\title{Future interpretation `without' future morphology}
\author{Jéssica + Aron (meeting \#1)}
\date{\today}

\begin{document}
\maketitle

\section{The project}

Some environments give rise to a non-scheduled future reading without \textit{will} or \textit{going to}. I don't really have a name for the phenomenon, so I've been calling it \textbf{future-oriented present (FOP)}, but this terminology is inaccurate\textemdash in some languages, you don't see \textsc{pres} morphology.

\pex\label{finite} 
\a\label{pred}If \textit{John wins the election}, Mary will be angry.\hfill [conditional antecedents]
\a Every politician \textit{who wins the election} will have to go through a background check.\hfill [some RCs]
\a When \textit{John wins the election}, Mary will be angry.\hfill [\textit{when}-clauses]
\a $\ldots$\footnote{This is a very sloppy presentation, if we were going to discuss the distribution of this phenomenon more seriously, we would have to control for the presence of \textit{will} in the sentence. In some cases, it seems that this future-oriented present is only conditionally licensed if \textit{will} is in the sentence.}
\xe

\ex\ljudge{*}John wins the election.\hfill [matrix clauses]\xe

%\pex\label{nonfinite} 
%\a John has \textit{to win the election}.\hfill [under some modals]
%\a John hopes \textit{to win the election}.\hfill [under some attitude verbs]
%\xe

I've been breaking down the project into three parts, but this division is somewhat artificial:

\begin{description}
    \item[The syntactic question] What's the LF of the FOP?
    \begin{itemize}
        \item Does it involve a \textsc{non-past}, \textsc{pres} + a covert forward-shifting morpheme, a deleted \textit{will}, or something else?
        \item A related question: how much in common does FOP have with future-oriented non-finite complements (i.e. modal complements, a.o.).
    \end{itemize}
    \item[The semantic question] What does the FOP mean?
    \begin{itemize}
        \item Is it purely temporal, or does it involve a layer of modality too? 
        \item How should each of these elements be analyzed?
    \end{itemize}
    \item[The distributional question] Why is the FOP restricted to some environments?
    \begin{itemize}
        \item I started from this question, but eventually I realized that it depends on the answer to the others.
        \item The received wisdom is that the FOP has a restricted distribution because of pragmatic principles regulating future reference. I think we should at least make an attempt to treat it as a case of polarity sensitivity.
    \end{itemize}
\end{description}

\section{The syntactic question}

\subsection{Previous approaches}

\paragraph{Approach \#1: FOP is a tense}A few authors treat the subjunctive future as a \textbf{\textsc{non-past}} \citep{schulz2008non,kaufmann2005conditional}. For these authors, predictive conditionals like \eqref{pred} belong in the same category as other indicative conditionals, like \eqref{ind}.

\ex\label{ind}If John won the election, Mary got angry.
\xe

\paragraph{Approach \#2: FOP is \textit{will}}The earlier literature treated the FOP as a \textbf{deleted \textit{will}} \citep{mccawley1971tense, comrie1985tense}. For them, \eqref{pred} is equivalent to \eqref{willcon}.

\ex\label{willcon}If John will win the election, Mary will get angry.
\xe

\paragraph{Approach \#3: FOP is aspect}In a more recent paper, \cite{williamson2021worlds} argues that the FOP is \textsc{pres + fut}, where \textsc{fut} is a \textbf{covert aspectual operator} (similar to \textsc{perf}). For him, \eqref{pred} is not the same as \eqref{ind} or \eqref{willcon}. Rather, it's the mirror image of \eqref{perf}.

\ex\label{perf}If John has gone to the party, Mary got angry.\xe

\subsection{My approach}

\begin{fancybox}{}
    The future-oriented present is a subjunctive future. That is, it has more in common with counterfactuals than with indicative conditionals.
\end{fancybox}

%I wanna argue that \eqref{pred} has more in common with subjunctive conditionals (i.e.: counterfactuals) than with \eqref{ind} or \eqref{willcon}.

%\ex\label{counter}If John went to the party, Mary would be happy.\xe

Here are some raw arguments to support this view. My plan is to flesh out these arguments, hopefully before the SALT deadline.

\paragraph{Argument \#1: morphological} In Brazilian Portuguese (BrP), to my knowledge, the only Romance language that retained a productive Subjunctive Future (\textsc{SFut}), \textsc{SFut} morphology appears in roughly the same environments where English licenses the FOP.

\ex\label{brp1}
\begingl 
\gla Se o João \textbf{for} \`{a} festa, a Maria pode ir também.//
\glb if the John \textbf{go.\textsc{subj.fut}} to.the party, the Mary might go too//
\glft `If John goes to the party, Mary might go too.'//
\endgl
\xe 

Crucially, the \sfut does not mean the same as the present. We can create minimal pairs with them.

\pex
\a\label{minsf}
\begingl 
\gla Se o João \textbf{viajar} pra Europa, a Maria deve ir também.//
\glb if the John \textbf{travel.\textsc{subj.fut}} to.the Europe, the Mary {is likely to} go too//
\glft `If John travels to Europe, Mary is likely to go too.'//
\endgl
\a\label{minpres}
\begingl 
\gla Se o João \textbf{viaja} pra Europa, a Maria deve ir também.//
\glb if the John \textbf{travel.\textsc{ind.pres}} to.the Europe, the Mary {is likely to} go too//
\glft `If John travels to Europe (habitually), Mary is likely to go too.'//
\endgl
\xe 

Example \eqref{minsf}, with the \sfut, is interpreted as a predictive conditional. \eqref{minpres}, with the \textsc{pres}, is interpreted as an epistemic conditional.

\begin{fancybox}{}
I need to prove that (1) the label `subjunctive future' is not misleading; (2) what's happening in BrP is also happening in English. I've been struggling with how to organize this argument.
\end{fancybox}

\paragraph{Argument \#2: clausal embedding} In many languages, \textit{if}-clauses can serve as embedded polar questions. It seems that this is only possible with \textit{indicative conditionals}. In this respect, FOP patterns with counterfactuals. 

\textbf{Note:} This is extremely difficult/impossible to see in English. You have to try really hard to block a scheduled reading of the present.

\pex 
\a\ljudge{?}I don't know whether John is elected (next year).\hfill [FOP]
\a\ljudge{?}I don't know whether John were elected (next year).\hfill [Counterfactual]
\xe

\pex 
\a\ljudge{?}I wonder whether John is elected (next year).\hfill [FOP]
\a\ljudge{?}I wonder whether John were elected (next year).\hfill [Counterfactual]
\xe

\pex 
\a I don't know whether John is home (now).\hfill [Indicative present]
\a I don't know whether John was elected.\hfill [Indicative past]
\a I don't know whether John will be elected.\hfill [\textit{will}]
\xe

\pex 
\a I wonder whether John is home (now).\hfill [Indicative present]
\a I wonder whether John was elected.\hfill [Indicative past]
\a I wonder whether John will be elected.\hfill [\textit{will}]
\xe

In BrP, the contrast is extremely clear.

\pex 
\a\ljudge{*}\begingl 
\gla Eu não sei se o João \textbf{for} eleito.//
\glb I \textsc{neg} know whether the João \textbf{be.\sfut} elected//
\glft `I don't know whether John is elected (in the future).'\hfill [FOP]//
\endgl
\a\ljudge{*}\begingl 
\gla Eu não sei se o João \textbf{fosse} eleito.//
\glb I \textsc{neg} know whether the João \textbf{be.\textsc{SPast}} elected//
\glft `I don't know whether John were elected (in the future).'\hfill [Counterfactual]//
\endgl
\xe

\pex 
\a\begingl 
\gla Eu não sei se o João está em casa.//
\glb I \textsc{neg} know whether the João \textbf{be.\textsc{ind.pres}} at home//
\glft `I don't know whether John is home.'\hfill [Indicative present]//
\endgl
\a\begingl 
\gla Eu não sei se o João estava em casa.//
\glb I \textsc{neg} know whether the João \textbf{be.\textsc{ind.past}} at home//
\glft `I don't know whether John was home.'\hfill [Indicative past]//
\endgl
\a\begingl 
\gla Eu não sei se o João vai estar em casa.//
\glb I \textsc{neg} know whether the João will be home//
\glft `I don't know whether John will be home.'\hfill [\textit{Will}]//
\endgl
\xe

\begin{fancybox}{}
I don't know much about clausal embedding, so I don't know how to explain these facts, or even if they are trite.
\end{fancybox}

\paragraph{Argument \#3: inversions???} This is extremely handwavy, and also a very weak argument, but I \textit{think} only subjunctive if-clauses allow inversions.

\pex
\a If John had come to the party, Mary would've been happy.\hfill [Counterfactual]
\a Had John come to the party, Mary would've been happy.
\xe 

\pex
\a If you need any help, let me know.\hfill [FOP]
\a Should you need any help, let me know.
\xe 

\pex 
\a If John is home (now), Mary is home too.\hfill [Indicative present]
\a\ljudge{?}Should John be home (now), Mary is home too.
\xe 

\pex 
\a If John left, Mary left too.\hfill [Indicative past]
\a\ljudge{?}Had John left, Mary left too.
\xe 

\begin{fancybox}{}
I like this argument because I know there is a bit of literature on the relation between I-to-C movement and the subjunctive. \textit{But} the judgments are subtle, and this kind of inversion is just really unusual anyway. I don't know if any reviewers would take it seriously.
\end{fancybox}

\section{The semantic question}

\textbf{Note:} I don't think any authors have investigated the semantic question with any systematicity. So it's a bit unfair to compare my `proposal' to theirs.

\begin{fancybox}{}
    The future-oriented present/subjunctive future invokes modal displacement, in addition to temporal displacement. %In terms of modal force, I believe it should be modeled as an indefinite.
\end{fancybox}

\paragraph{Evidence} The \sfut in relative clauses seems to block a specific reading of the modified noun.

\pex 
\a Every student who gets an offer from our department (next year) will be fully funded.
\a $\not\leadsto$ We know which students will get an offer.
\xe 

\pex 
\a Every student who will get an offer from our department (next year) will be fully funded.
\a $\leadsto$ We know which students will get an offer.
\xe 

In contexts that require a specific reading of the noun, the \sfut is infelicitous.

\pex\textbf{Context:} You're throwing a dinner party. Among the people who RSVP'd, everyone is a vegan. 
\a\ljudge{$\#$}Everyone who comes to the part is a vegan.
\a Everyone who will come to the party is a vegan.
\xe 



\bibliography{phd.bib}
\bibliographystyle{chicago}

\end{document}